\documentclass[oneside,a4paper,12pt]{article} % Specifies the page format and font size.

% -------------------------------------- Integration of packages -------------------------------------- 
\usepackage[utf8]{inputenc} % Enables the use of special characters.
% Literature and language
	\usepackage[english]{babel}
	\usepackage[style=bwl-FU,backend=bibtex,natbib=true,maxcitenames=2]{biblatex}
	\addbibresource{Content/bibliography.bib}
	\usepackage{csquotes}
% Format and layout
	\usepackage[left=3cm,right=3cm,bottom=3cm]{geometry} % Specifies left and right side margins.
	\usepackage{setspace} % Package that enables modifying the line spacing.
	\setstretch{1.3} % Sets a line spacing of 1.3.
	\parindent0pt % Sets the left indent at a new paragraph.
    \parskip10pt % Sets the space between two paragraphs.
	\usepackage{footmisc} % Implements a range of footnote options.
	\renewcommand{\footnotelayout}{\setstretch{1}} % Sets a line spacing of 1 for the footnotes.
	\pagestyle{headings} % Creates a header using the page number and the heading of the current section.
	\usepackage[T1]{fontenc} % Enables the use of special characters.
	\usepackage{eurosym} % Usage of €
	\usepackage{acronym} % Enables the incorporation of a list of abbreviations.
	\usepackage{nomencl} % Useful to create a list of symbols.
	\usepackage{enumerate} % Useful for enumerations.
	\usepackage{color} % Enables the definition of colors.
	
% Tables and Graphs
	\usepackage{booktabs} % Improves the design of the tables
	\usepackage{longtable} % Allows tables to be longer than one page.
	\usepackage{multirow,multicol} % With this package it is now possible to combine columns and rows within tables.
	\usepackage{graphicx} % Allows to implement graphics.
	\usepackage{subfig} % Enables graphs consisting of several figures.
	\graphicspath{{./Graphs/}} % Tells LATEX that the images are kept in a folder named images under the directory of the main document.
	\usepackage[hypcap=false]{caption} % Provides many ways to customize captions.	

% Mathematics
    \usepackage{amscd,amsfonts,amsmath,amssymb,amsthm,amscd,bbm} % Extends the math set.


% --------------------------------- Information on thesis --------------------------------- 
% Please fill in this information once at the beginning. This way, gaps will be filled in automatically in the following.
\newcommand{\name}{Markus Bilz} % Enter your name.
\newcommand{\titleofthesis}{Option Trade Classification using Machine Learning} % Enter the title of your thesis.
\newcommand{\streetadress}{Mathystr.~14-16 // XI-11} % Enter your street address.
\newcommand{\postalcode}{76133} % Enter your postal code.
\newcommand{\city}{Karlsruhe} % Enter your city/town.
\newcommand{\email}{mail@markusbilz.com} % Enter your email address.
\newcommand{\typeofthesis}{Master Thesis} % specify the type of thesis: Seminar Thesis, Bachelor Thesis, Master Thesis.
\newcommand{\dateofthesis}{\today}

% --------------------------------- Definition of hyperlinks --------------------------------- 
% Hyperreferences
	\usepackage{hyperref}
	\definecolor{darkblue}{rgb}{0,0,.5}
	\hypersetup{
	    pdfstartview={FitH}, 
        colorlinks=true,
        linkcolor=black,
        citecolor=darkblue,
        urlcolor=black,
        pdftitle=\titleofthesis,
        pdfsubject=\typeofthesis,
        pdfauthor=\name,
        bookmarksopen
        }

	\usepackage[acronym]{glossaries}
	\makeglossaries
			
	\newacronym{ANN}{ANN}{artificial neural network}
	\newacronym{AUC}{AUC}{area under the curve}
	\newacronym{CRSP}{CRSP}{Center for Research in Securities Prices}
	\newacronym{CBOE}{CBOE}{Chicago Board Options Exchange}
	\newacronym{CLNV}{CLNV}{Chakrabarty-Li-Nguyen-Van-Ness}
	\newacronym{EMO}{EMO}{Ellis-Michaely-O’Hara}
	\newacronym{FFN}{FFN}{feed-forward network}
	\newacronym{GBM}{GBM}{gradient boosting machine}
	\newacronym{ISE}{ISE}{International Securities Exchange}
	\newacronym{LR}{LR}{Lee-Ready}
	\newacronym[firstplural=long short-term memories (LSTMs)]{LSTM}{LSTM}{long short-term memory}
	\newacronym{MAE}{MAE}{mean absolute error}
	\newacronym{MSE}{MSE}{mean squared error}
	\newacronym{RMSE}{RMSE}{root mean squared error}
	\newacronym{RF}{RF}{random forest}
	\newacronym{SSE}{SSE}{sum of squared errors}
	\newacronym{SHAP}{SHAP}{SHapley Additive exPlanations}		
	

% ----------------------------------- Start of document ----------------------------------- 
\begin{document}
\setcounter{page}{2} % Cover pages and title page are not numbered. Start numbering from page 2.

% Title page 
%	\newgeometry{left=3cm, right=3cm, bottom=2cm}
\begin{titlepage}
		\begin{center}
			{\Large Karlsruhe Institute of Technology \\
			\vspace{0.6cm}
			Institute for Finance\\
			Department of Financial Engineering and Derivatives\\
			Prof. Dr. Marliese Uhrig-Homburg} \\[4.5cm]
			{\large{\typeofthesis}} \\[1.7cm]
			\setstretch{10.0}
			{\Huge {\titleofthesis}}
			\setstretch{1.3} \\[7cm]
		\end{center}
				
		\begin{tabular}{ll}
        Author:     & {\name}\\
                    & {\streetadress}\\
                    & {\postalcode} {\city}\\
					& E-Mail: {\email}\\\\
        Karlsruhe, & {\dateofthesis}\\
    	\end{tabular}
\end{titlepage}
\restoregeometry % Exclude title page (with %) that is not being used.
\newgeometry{left=3cm, right=3cm, bottom=2cm}
\begin{titlepage}
		\begin{center}
			{\Large Karlsruhe Institute of Technology \\
			\vspace{0.6cm}
			Institute for Finance\\
			Department of Financial Engineering and Derivatives\\
			Prof. Dr. Marliese Uhrig-Homburg} \\[4.5cm]
			{\large{\typeofthesis}} \\[1.7cm]
			\setstretch{10.0}
			{\Huge {\titleofthesis}}
			\setstretch{1.3} \\[7cm]
		\end{center}
				
		\begin{tabular}{ll}
        Author:     & {\name}\\
                    & {\streetadress}\\
                    & {\postalcode} {\city}\\
					& E-Mail: {\email}\\\\
        Karlsruhe, & {\dateofthesis}\\
    	\end{tabular}
\end{titlepage}
\restoregeometry
% Main text section
\newpage
\setcounter{page}{1}\renewcommand{\thepage}{\arabic{page}} % Sets the numbering to Arabic.


\section{Background and Motivation}

Every option trade has a buyer and seller side. For a plethora of problems in option research, it’s also crucial to determine the party that initiated the transaction. Applications include the study of option demand \autocite[][]{garleanuDemandBasedOptionPricing2009}, of informational content in option trading \autocites[][]{huDoesOptionTrading2014}[][]{panInformationOptionVolume2006}[][]{caoInformationalContentOption2005}, of order flow \autocite[][]{muravyevOrderFlowExpected2016}, or of trading costs \autocite[][]{muravyevOptionsTradingCosts2020}. 

Despite the clear importance for empirical research, the true initiator of the trade is frequently absent in datasets and is inferred using trade classification rules \autocite[][]{easleyOptionVolumeStock1998}. In consequence, the correctness of empirical studies hinges on the algorithm's ability to accurately identify the trade initiator.

Popular heuristic to sign trades are the tick test \autocite[][]{hasbrouckTradesQuotesInventories1988}, quote rule \autocite[][]{harrisDayEndTransactionPrice1989}, and hybrids thereof such as the \gls{LR} algorithm \autocite[][]{leeInferringTradeDirection1991}. These rules have initially been proposed and tested in the stock market. For option markets, the works of \textcites[][]{savickasInferringDirectionOption2003}[][]{grauerOptionTradeClassification2022} raise concerns about the transferability of trade signing rules due to deteriorating classification accuracies and systematic misclassifications. The latter is crutial, as non-random misclassifications bias the dependent research \autocites[][]{odders-whiteOccurrenceConsequencesInaccurate2000}[][]{theissenTestAccuracyLee2001}.

A contending body of research \autocites{blazejewskiLocalNonParametricModel2005}{rosenthalModelingTradeDirection2012}{ronenMachineLearningTrade2022} improves trade classification performance through \gls{ML}. The scope of current research is yet bound to the stock market and the \textit{superficial} setting, where fully-labelled trades are available. 

The goal of our empirical study is to investigate if machine learning-based classifier improve upon the accuracy of state-of-the-art approaches for option trade classification?

\section{Contributions}

Our contributions are three-fold: 
\begin{enumerate}[label=(\roman*),noitemsep]

\item By employing gradient-boosted trees and transformers we are able to establish a new state-of-the-art in terms of classification accuracy. We outperform existing approaches by (...) in accuracy with comparable data requirements. We show that the model generalizes on other exchanges (...) Relative to the ubiquitous \gls{LR} algorithm, improvements are between (...) and (...).
\item In practice, unlabelled trades are abundant, whereas true labels for trades are scarce. Our work is the first to consider both the supervised and the semi-supervised setting, where trades are only required to be partially-labelled.
\item Through a feature importance analysis based on Shapley values, we can consistently attribute performance gains of rule-based and \gls{ML}-based classifiers to feature groups. We discover that both paradigms share features to a large extent, but \gls{ML}-based approaches more effectively exploit the data.
\end{enumerate}

\section{Data}

% We trained on the standard WMT 2014 English-German dataset consisting of about 4.5 million sentence pairs. Sentences were encoded using byte-pair encoding [3], which has a shared sourcetarget vocabulary of about 37000 tokens.

We perform the empirical analysis on two large-scale datasets of option trades recorded at the \gls{ISE} and \gls{CBOE}. Our sample construction follows \textcite[][]{grauerOptionTradeClassification2022}, which fosters comparability between both works. 

After a time-based train-validation-test split (60-20-20), required by the \gls{ML} estimators, we are left with two test set spanning from November 2015 -- May 2017 at the \gls{ISE} and Nov. 2015 -- Oct. 2017 at the \gls{CBOE}, respectively. Each test set contains between 9.8 Mio. --  12.8 Mio. labelled option trades. An additional unlabelled, training set of \gls{ISE} trades executed between Oct. 2012 -- Oct. 2013 is reserved for learning in the semi-supervised setting.

To establish a common ground with rule-based classification, we distinguish three feature sets with increasing data requirements and employ minimal feature engineering. The first set is based on the data requirements of tick/quote-based algorithms, the second of hybrid algorithms with additional dependencies on trade size data, such as the \gls{GSU} method, and the third feature set includes option characteristics, like the option's $\Delta$ or the underlying. 

\section{Methodology}

We model trade classification using gradient-boosted trees \autocites[][]{friedmanGreedyFunctionApproximation2001}, a wide tree-based ensemble, and the FT-Transformer \autocite{gorishniyRevisitingDeepLearning2021}, a Transformer-based neural network architecture. We chose these approaches for their state-of-the-art performance in tabular modelling \autocites[][]{gorishniyRevisitingDeepLearning2021}[][]{grinsztajnWhyTreebasedModels2022} and their extendability to learn on partially-labelled trades. Additionally, Transformers offer \textit{some} model interpretability through the Attention mechanism. An advantage we exploit later to derive insights into the decision process of Transformers.

As stated earlier, our goal is to extend \gls{ML} classifiers for the semi-supervised setting to make use of the abundant, unlabelled trade data. We couple gradient-boosting with self-training \autocite{yarowskyUnsupervisedWordSense1995}, whereby confident predictions of unlabelled trades are interatively added into the training set as pseudo-labels. A new classifier is then retrained on labelled and pseudo-labelled trades. Likewise, the Transformer is pre-trained on unlabelled trades with the replaced token detection objective of \textcite{clarkElectraPretrainingText2020} and later finetuned on labelled training instances. Conceptually, the network learns to detect randomly replaced tokens or features of transactions. Both techniques are aimed at improving generalization performance.

Classical trade classification rules are implemented as rule-based classifier allowing us to construct arbitrary candidates for benchmarking and support richer evaluation of feature importances.\footnote{The implementation is publically available under \url{https://pypi.org/project/tclf/}.}

To facilitate a fair comparison, we run an exhaustive Bayesian search, to find a suitable hyperparameter configuration for each of our models. Classical
rule have no hyperparameters per se. Akin to tuning the machine learning classifiers on the validation set, we select the classical benchmarks based on their validation performance. This is most rigorous, while preventing to overfit the test set.\footnote{All of our source code and experiments are publically available under \url{https://github.com/KarelZe/thesis}.}

\section{Results}

Following good measures, we perform robustness tests across different sub-samples such as option type, type of underlying, time among others. 


% \section{Relevancy}


% Bibliography
\newpage
\printbibliography
\end{document}