\documentclass[oneside,a4paper,10pt]{article} % Specifies the page format and font size.

% -------------------------------------- Integration of packages -------------------------------------- 
\usepackage[utf8]{inputenc} % Enables the use of special characters.
% Literature and language
	\usepackage[english]{babel}
	\usepackage[style=bwl-FU,backend=bibtex,natbib=true,maxcitenames=2]{biblatex}
	\addbibresource{Content/bibliography.bib}
	\usepackage{csquotes}
% Format and layout
	\usepackage[left=3cm,right=3cm,bottom=3cm]{geometry} % Specifies left and right side margins.
	\usepackage{setspace} % Package that enables modifying the line spacing.
	\setstretch{1.3} % Sets a line spacing of 1.3.
	\parindent0pt % Sets the left indent at a new paragraph.
    \parskip10pt % Sets the space between two paragraphs.
	\usepackage{footmisc} % Implements a range of footnote options.
	\renewcommand{\footnotelayout}{\setstretch{1}} % Sets a line spacing of 1 for the footnotes.
	\pagestyle{headings} % Creates a header using the page number and the heading of the current section.
	\usepackage[T1]{fontenc} % Enables the use of special characters.
	\usepackage{eurosym} % Usage of €
	\usepackage{acronym} % Enables the incorporation of a list of abbreviations.
	\usepackage{nomencl} % Useful to create a list of symbols.
	\usepackage{enumerate} % Useful for enumerations.
	\usepackage{color} % Enables the definition of colors.
	
% Tables and Graphs
	\usepackage{booktabs} % Improves the design of the tables
	\usepackage{makecell} %Connected rows
	\usepackage{longtable} % Allows tables to be longer than one page.
	\usepackage{multirow,multicol} % With this package it is now possible to combine columns and rows within tables.
	\usepackage{graphicx} % Allows to implement graphics.
	\usepackage{subfig} % Enables graphs consisting of several figures.
	\graphicspath{{./Graphs/}} % Tells LATEX that the images are kept in a folder named images under the directory of the main document.
	\usepackage[hypcap=false]{caption} % Provides many ways to customize captions.
	
	\usepackage{enumitem} % enumerate with letters https://tex.stackexchange.com/a/129960

% Mathematics
    \usepackage{amscd,amsfonts,amsmath,amssymb,amsthm,amscd,bbm} % Extends the math set.


	\usepackage{siunitx} % Enables the use of SI units e. g., proper handling of percentage

	% manually define opening bracket which is otherwise parsed by sunitx
	% https://tex.stackexchange.com/q/450026/169093
	\usepackage{etoolbox} 
	\newrobustcmd{\parl}{(}
	\newrobustcmd{\parr}{)}
	
	\newcommand*{\tabindent}{ \hspace{2mm}} % Indentation for tables
	
	\newcommand{\todo}[1]{\textcolor{red}{TODO: #1}\PackageWarning{TODO:}{#1!}} % Todos
	
	\sisetup{round-mode=places,round-precision=2, group-separator={,},output-decimal-marker={.}, round-pad = false, input-symbols = {(-)}, % separate-uncertainty=false,
	group-minimum-digits=4, % 1,000 instead of 1000
	% table-space-text-pre={(},
	% table-align-text-pre=false,
	% table-space-text-post={$^{***}$},
	table-align-text-post=false,
	detect-weight=true,
	detect-inline-weight=math,
	retain-explicit-plus=true,
	% round-integer-to-decimal=false,
	round-precision=2,
	table-format = 1.3, 
	output-exponent-marker=\ensuremath{\mathrm{e}}
	} % round to 2 decimal places

% --------------------------------- Information on thesis --------------------------------- 
% Please fill in this information once at the beginning. This way, gaps will be filled in automatically in the following.
\newcommand{\name}{Markus Bilz} % Enter your name.
\newcommand{\titleofthesis}{Improving Option Trade Classification using Machine Learning} % Enter the title of your thesis.
\newcommand{\streetadress}{Mathystr.~14-16 // XI-11} % Enter your street address.
\newcommand{\postalcode}{76133} % Enter your postal code.
\newcommand{\city}{Karlsruhe} % Enter your city/town.
\newcommand{\email}{mail@markusbilz.com} % Enter your email address.
\newcommand{\typeofthesis}{Master Thesis} % specify the type of thesis: Seminar Thesis, Bachelor Thesis, Master Thesis.
\newcommand{\dateofthesis}{\today}

% --------------------------------- Definition of hyperlinks --------------------------------- 
% Hyperreferences
	\usepackage{hyperref}
	\definecolor{darkblue}{rgb}{0,0,.5}
	\hypersetup{
	    pdfstartview={FitH}, 
        colorlinks=true,
        linkcolor=black,
        citecolor=darkblue,
        urlcolor=black,
        pdftitle=\titleofthesis,
        pdfsubject=\typeofthesis,
        pdfauthor=\name,
        bookmarksopen
        }


	\usepackage[capitalise, noabbrev]{cleveref} % Enables the use of \cref{} to refer to figures, etc. with Fig.
	\creflabelformat{equation}{#2#1#3} % omit round brackets
	% https://tex.stackexchange.com/a/121055/169093
	\Crefname{appsec}{appendix}{appendices}

	\usepackage[acronym]{glossaries}
	\makeglossaries
			
	\newacronym{ANN}{ANN}{artificial neural network}
	\newacronym{AUC}{AUC}{area under the curve}
	\newacronym{CRSP}{CRSP}{Center for Research in Securities Prices}
	\newacronym{CBOE}{CBOE}{Chicago Board Options Exchange}
	\newacronym{CLNV}{CLNV}{Chakrabarty-Li-Nguyen-Van-Ness}
	\newacronym{EMO}{EMO}{Ellis-Michaely-O’Hara}
	\newacronym{FFN}{FFN}{feed-forward network}
	\newacronym{GBM}{GBM}{gradient boosting machine}
	\newacronym{GSU}{GSU}{Grauer-Schuster-Uhrig-Homburg}
	\newacronym{GBRT}{GBRT}{gradient-boosted regression tree}
	\newacronym{ISE}{ISE}{International Securities Exchange}
	\newacronym{LR}{LR}{Lee-Ready}
	\newacronym[firstplural=long short-term memories (LSTMs)]{LSTM}{LSTM}{long short-term memory}
	\newacronym{MAE}{MAE}{mean absolute error}
	\newacronym{MSE}{MSE}{mean squared error}
	\newacronym{NBBO}{NBBO}{national best bid and offer}
	\newacronym{ML}{ML}{machine learning}
	\newacronym{RMSE}{RMSE}{root mean squared error}
	\newacronym{RF}{RF}{random forest}
	\newacronym{SSE}{SSE}{sum of squared errors}
	\newacronym{SHAP}{SHAP}{SHapley Additive exPlanations}
	\newacronym{SAGE}{SAGE}{Shapley Additive Global importancE}		
	

% ----------------------------------- Start of document ----------------------------------- 
\begin{document}
\setcounter{page}{2} % Cover pages and title page are not numbered. Start numbering from page 2.

% Title page 
%	\newgeometry{left=3cm, right=3cm, bottom=2cm}
\begin{titlepage}
		\begin{center}
			{\Large Karlsruhe Institute of Technology \\
			\vspace{0.6cm}
			Institute for Finance\\
			Department of Financial Engineering und Derivatives\\
			Prof. Dr. Marliese Uhrig-Homburg} \\[4.5cm]
			{\large{\typeofthesis}} \\[1.7cm]
			\setstretch{10.0}
			{\Huge {\titleofthesis}}
			\setstretch{1.3} \\[7cm]
		\end{center}
				
		\begin{tabular}{ll}
        Author:     & {\name}\\
                    & {\streetadress}\\
                    & {\postalcode} {\city}\\
					& E-Mail: {\email}\\\\
        Karlsruhe, & {\dateofthesis}\\
    	\end{tabular}
\end{titlepage}
\restoregeometry % Exclude title page (with %) that is not being used.
\newgeometry{left=3cm, right=3cm, bottom=2cm}
\begin{titlepage}
		\begin{center}
			{\Large Karlsruhe Institute of Technology \\
			\vspace{0.6cm}
			Institute for Finance\\
			Department of Financial Engineering und Derivatives\\
			Prof. Dr. Marliese Uhrig-Homburg} \\[4.5cm]
			{\large{\typeofthesis}} \\[1.7cm]
			\setstretch{10.0}
			{\Huge {\titleofthesis}}
			\setstretch{1.3} \\[7cm]
		\end{center}
				
		\begin{tabular}{ll}
        Author:     & {\name}\\
                    & {\streetadress}\\
                    & {\postalcode} {\city}\\
					& E-Mail: {\email}\\\\
        Karlsruhe, & {\dateofthesis}\\
    	\end{tabular}
\end{titlepage}
\restoregeometry
% Main text section
\newpage
\setcounter{page}{1}\renewcommand{\thepage}{\arabic{page}} % Sets the numbering to Arabic.

% The dominant sequence transduction models are based on complex recurrent or
% convolutional neural networks that include an encoder and a decoder. The best
% performing models also connect the encoder and decoder through an attention
% mechanism. We propose a new simple network architecture, the Transformer,
% based solely on attention mechanisms, dispensing with recurrence and convolutions
% entirely. Experiments on two machine translation tasks show these models to
% be superior in quality while being more parallelizable and requiring significantly
% less time to train. Our model achieves 28.4 BLEU on the WMT 2014 Englishto-German translation task, improving over the existing best results, including
% ensembles, by over 2 BLEU. On the WMT 2014 English-to-French translation task,
% our model establishes a new single-model state-of-the-art BLEU score of 41.8 after
% training for 3.5 days on eight GPUs, a small fraction of the training costs of the
% best models from the literature. We show that the Transformer generalizes well to
% other tasks by applying it successfully to English constituency parsing both with
% large and limited training data

\section{Background and Motivation}

Every option trade has a buyer and seller side. For a plethora of problems in option research, it’s also crucial to determine the party that initiated the transaction. Applications include the study of option demand \autocite[][]{garleanuDemandBasedOptionPricing2009}, of informational content in option trading \autocites[][]{huDoesOptionTrading2014}[][]{panInformationOptionVolume2006}[][]{caoInformationalContentOption2005}, of order flow \autocite[][]{muravyevOrderFlowExpected2016}, or of trading costs \autocite[][]{muravyevOptionsTradingCosts2020}. 

Despite the clear importance for empirical research, the true initiator of the trade is frequently absent in datasets and is inferred using trade classification rules \autocite[][]{easleyOptionVolumeStock1998}. In consequence, the correctness of empirical studies hinges on the algorithm's ability to accurately identify the trade initiator.

Popular heuristic to sign trades are the tick test \autocite[][]{hasbrouckTradesQuotesInventories1988}, quote rule \autocite[][]{harrisDayEndTransactionPrice1989}, and hybrids thereof such as the \gls{LR} algorithm \autocite[][]{leeInferringTradeDirection1991}. These rules have initially been proposed and tested in the stock market. For option markets, the works of \textcites[][]{savickasInferringDirectionOption2003}[][]{grauerOptionTradeClassification2022} raise concerns about the transferability of trade signing rules due to deteriorating classification accuracies and systematic misclassifications. The latter is crutial, as non-random misclassifications bias the dependent research \autocites[][]{odders-whiteOccurrenceConsequencesInaccurate2000}[][]{theissenTestAccuracyLee2001}.

A contending body of research \autocites{blazejewskiLocalNonParametricModel2005}{rosenthalModelingTradeDirection2012}{ronenMachineLearningTrade2022} improves trade classification performance through \gls{ML}. The scope of current research is yet bound to the stock market and the \textit{artificial} setting, where fully-labelled trades are available. 

The goal of our empirical study is to investigate if machine learning-based classifier improve upon the accuracy of state-of-the-art approaches for option trade classification?

\section{Contributions}

% Thereby, our work addresses several addressed shortcomings.
% TODO: by how much? 
% Our approaches outperform all rule-based approaches on International Securities Exchange (ISE) and Chicago Board Options Exchange (CBOE) data with comparable data requirements.

Our contributions are three-fold: 
\begin{enumerate}[label=(\roman*),noitemsep]
\item By employing gradient-boosted trees and transformers we are able to establish a new state-of-the-art in terms of classification accuracy. 
\item Our work is the first to consider both the supervised and the semi-supervised setting, where trades are partially-labelled.
\item Through a feature importance analysis based on Shapley values, we can consistently attribute performance gains of rule-based and \gls{ML}-based classifiers to feature groups. We discover that both paradigms share common features, but \gls{ML}-based approaches more effectively exploit the data. % Additional insights are gained from probing the Transformers' attention heads.
\end{enumerate}

% consistently attribute probing attention heads

% Our We perform rigorous benchmarking.

% rendering irrelevant


% We assess the performance
% Wee apply and where only a small fraction of trades can be labelled and peform a regorous benchmarking.


% In summary, machine learning has been applied successfully in the context of trade
% classification. A summary is given in Appendix A.1. No previous work performs
% machine learning-based classification in the options markets. Our work fills this gap
% and models trade classification using machine learning to improve upon extant rules.



% The dataset is split into three disjoint sets for training, validation, and testing. As in \textcite{ellisAccuracyTradeClassification2000} and \textcite{ronenMachineLearningTrade2022} we perform a classical train-test split, thereby maintaining the temporal ordering within the data.

% So far research focuses on the  None of which have been tested
% hier ml based algos beschreiben


% TODO: case of partially labelled dat + 
% TODO: Hier überleitung ML einführen.
% Envolving




% Recent works 

% Recent work of \textcite[][13--16]{grauerOptionTradeClassification2022} made significant progress in classification accuracy by proposing explicit overrides for order types and by combining multiple heuristics, thereby advancing the state-of-the-art performance in option trade classification. By this means, their approach enforces a more sophisticated decision boundary eventually leading to a more accurate classification. The fundamental constraint is, that overrides apply only to subsets of trades. Beyond heuristics, it remains open, if classifiers \emph{learned} on trade data can improve upon \emph{static} classification rules in terms of performance and robustness.


\section{Data}

% We trained on the standard WMT 2014 English-German dataset consisting of about 4.5 million sentence pairs. Sentences were encoded using byte-pair encoding [3], which has a shared sourcetarget vocabulary of about 37000 tokens.

We perform the empirical analysis on two large-scale datasets of option trades recorded at the \gls{ISE} and \gls{CBOE}. Our sample construction follows \textcite[][]{grauerOptionTradeClassification2022}, which fosters comparability between both works. 

After a time-based train-validation-test split (60-20-20), required by the \gls{ML} estimators, we are left with two test set spanning from November 2015 -- May 2017 at the \gls{ISE} and November 2015 -- October 2017 at the \gls{CBOE}, respectively. Each test set contains between 9.8 Mio. --  12.8 Mio. labelled option trades. An additional unlabelled, training set of \gls{ISE} trades executed between Oct. 2012 -- Oct. 2013 is reserved for semi-supervised learning.

To establish a common ground with rule-based classification, we distinguish three feature sets with increasing data requirements and apply minimal feature engineering. The first set is based on the data requirements of tick/quote-based algorithms, the second of complex, hybrid algorithms with additional dependencies on trade size data, such as the \gls{GSU} method, and the third feature set includes option characteristics, like the option's $\Delta$. 

\section{Methodology}

We employ ML technology for trade classification, namely gradient-boosted trees, a wide tree-based ensemble, and the FT-Transformer \autocite{gorishniyRevisitingDeepLearning2021}, an Attention-based neural network architecture. We chose these approaches for their state-of-the-art classification performance in tabular modelling \autocites[][]{gorishniyRevisitingDeepLearning2021}[][]{grinsztajnWhyTreebasedModels2022} and their extendability to learn on partially-labelled trades.



We consider 


We implement all rule-based approachs in 

% https://github.com/KarelZe/tclf

W

Analogous to tuning ML models on the validation set, we select the benchmarks on the validation set. For 


To facilitate a fair comparison, we sele


\section{Results}

Following good measures, we perform robustness tests across different sub-samples such as option type, type of underlying, time among others. 


% \section{Relevancy}


% Bibliography
\newpage
\printbibliography
\end{document}