\section{Related Work}\label{sec:related-work}

This chapter introduces research on trade classification in option markets and machine learning-based trade classification.

\subsection{Trade Classification in Option Markets}
\label{sec:trade-classification-in-option-markets}

While classical trade classification algorithms are extensively tested in the stock markets (e.g., \textcite[][3806--3821]{chakrabartyTradeClassificationAlgorithms2012}; \textcite[][259--286]{odders-whiteOccurrenceConsequencesInaccurate2000}), few works have examined trade classification in option markets.

\textcite[882--883]{savickasInferringDirectionOption2003} were the first to compare the tick rule, quote rule, the \gls{LR} algorithm and the \gls{EMO} rule for options traded at the \gls{CBOE}. The dataset spans a period from July 3, 1995 -- December 31, 1995 consisting of $869{,}217$ matched trades. The authors report the highest accuracies for the quote rule (\SI{78.98}{\percent}) and find that all rules perform worst when applied to index options. In general, the trade classification rules exhibit significantly lower classification accuracies on options data than with stock data, urging the need for improved classifiers.

The most exhaustive study is the one of \textcite[1--39]{grauerOptionTradeClassification2022}. The authors test the accuracy of the classical quote rule and tick rule, and hybrids thereof on two large-scale datasets spanning a period from 2005 to 2017~\footnote{We formally define accuracy in \cref{sec:evaluation-metric}.}. Consistently for options traded at the \gls{CBOE} and \gls{ISE} classical rules like the popular \gls{LR}  algorithm only achieve accuracies of \SI{62.03}{\percent} or \SI{62.53}{\percent} and are thus significantly smaller than in the stock market. In line with the research of \textcite[886]{savickasInferringDirectionOption2003}, the reported accuracies are inversely proportional to the rule's reliance on past transaction prices. In particular, the tick rule performs worst with accuracies marginally different from a random guess. Overall, the success rates of classical algorithms deteriorates between both studies and over time. As a remedy, \textcite[14--17]{grauerOptionTradeClassification2022} introduce two additional rules based on the trade and quote sizes. The depth rule is an alternative to the tick rule for classifying midspread trades in the \gls{LR}  algorithm and the \gls{EMO} rule. It assigns the initiator of the trade based on the depth at the bid or ask. Together with the trade size rule, their second rule, which classifies trades with a trade size matching the size of the bid or ask quote, can substantially improve the performance of classical algorithms. The ensemble of rules achieves an accuracy between \SI{73}{\percent} and \SI{75}{\percent} surpassing previous rules by more than \SI{10}{\percent}.

\todo{Think of better transition e.g., our analysis are based on the same datasets, we take rules as  a benchmark.}

The work of \textcite[1--39]{grauerOptionTradeClassification2022} is relevant for this thesis for two reasons. First, the dataset is identical to ours, which enables a fair comparison between classical rules and machine learning-based predictors. Second, their stacked combinations of the trade size rule, depth rule, and common trade classification algorithms achieve state-of-the-art performance in option trade classification and are thus a rigorous benchmark.

\subsection{Trade Classification Using Machine Learning}
\label{sec:trade-classification-using-machine-learning}

\textcite[5]{rosenthalModelingTradeDirection2012} bridges the gap between classical trade classification and machine learning by fitting a logistic regression model on lagged and unlagged features innate to the tick rule, quote rule, and \gls{EMO} algorithm, as well as a sector-specific and a time-specific term. Instead of using the rule's discretized outcome as a feature, he models the rules through so-called information strength functions \autocite[6--7]{rosenthalModelingTradeDirection2012}. The proximity to the quotes, central to the \gls{EMO} algorithm, is thus modelled by a proximity function. Likewise, the information strength of the quote and tick rule is estimated as the log return between the trade price and the midpoint or the previous trade price. However, it only improves the accuracy of the \gls{EMO} algorithm by a marginal \SI{2.0}{\percent} for \gls{NASDAQ} stocks and \SI{1.1}{\percent} for \gls{NYSE} stocks \autocite[15]{rosenthalModelingTradeDirection2012}. Our work aims to improve the model by exploring non-linear estimators and minimising data modelling assumptions.

The work of \textcite[483]{blazejewskiLocalNonParametricModel2005} compares a $k$-nearest neighbour classifier against logistic regression, as well as simple heuristics like the majority vote over past trades for signing trades at the Australian stock exchange. Their results indicate that the parametric $k$-nearest neighbour classifier improves upon a linear logistic regression in terms of classification accuracy, even when trained on fewer features. The work is unique from the remaining works about the feature set definition. Notably, \textcite[483]{blazejewskiLocalNonParametricModel2005} use no quote or trade prices, but rather the order book volumes, trade sizes, and past trade signs for classification. No accuracies for classical trade signing rules are reported, which impedes a comparison across different works. In line with their results, we focus on non-linear models in the form of gradient boosting and Transformers. Additionally, our paper addresses the mentioned shortcomings by benchmarking against state-of-the-art trade classification rules. We share the idea of using the trade size, as well as the bid and ask sizes for classification for some of our feature sets, but greedily predict using non-historic features.

Closest to our work is a publication by \textcite[1--58]{ronenMachineLearningTrade2022}. Therein, the authors compare a selection of machine learning algorithms against classical trade signing rules in the bond and stock market. Their comparison is the first to consider logistic regression, a random forest, as well as \glspl{feed-forward-network}. Over a wide range of feature sets the tree-based ensemble consistently outperforms by out-of-sample accuracy the tick rule and \gls{LR} algorithm, as well as all remaining machine learning models. For the \gls{TRACE} and ITCH dataset, their best variant of the random forest outperforms the tick rule by \SI{8.3}{\percent} and \SI{3.3}{\percent}, respectively \autocite[57]{ronenMachineLearningTrade2022}. Whilst the superiority of random forests is consistent for the bond and equity market, fitted classifiers do not transfer across markets, as accuracies diminish in a transfer setting.

The results convincingly demonstrate the potential of machine learning, i.e., of tree-based ensembles, for trade classification. Yet, the comparability of the results is limited by the classifier's reliance on additional features beyond quote and price data. Albeit, \textcite[13--14]{ronenMachineLearningTrade2022} consider a wide range of approaches, their selection leaves the latest advancements in artificial neural networks and ensemble learning aside and is mainly guided by computational constraints. Even if the focus is on standard techniques, the unclear research agenda concerning model selection, tuning, and testing hampers the transferability of their results to the yet unstudied option market.

In summary, machine learning has been applied successfully in the context of trade classification. No previous work performs, machine learning-based classification in the options markets.

\todo{Pick up research question. redundancy is ok here. We fill this gap and transfer ML to the options market to improve existing rules.}

\subsection{Research Framework}\label{sec:research-framework}

We present our research framework for trade classification, summarised in \cref{fig:research-framework}. Our approach revolves around two key ideas. First, we utilise \glspl{GBRT} and Transformers for trade classification, chosen in \cref{sec:supervised-approaches} for their expected performance, scalability, and extensibility. Distinctions are made between models trained on singly labelled trades and models trained on both labelled and unlabelled trades simultaneously. Second, classical trade classification rules, such as the \gls{LR}, are realised as a generic classifier leveraging the stacking principle, thereby enabling a coherent evaluation and model interpretation.

\begin{figure}[!ht]
    \centering
    {\renewcommand\normalsize{\tiny}
    \normalsize
    \input{./Graphs/research-framework.pdf_tex}}
    \caption[Research Framework]{Research Framework}
    \label{fig:research-framework}
\end{figure}

The data preparation process, outlined in \cref{sec:data-and-data-preparation}, encompasses all steps necessary to obtain features to be processed by the classifiers. Model enhancements, training setups, and tuning procedures are detailed in \cref{sec:training-and-tuning}. The predictions of the classifiers are consistently evaluated in terms of accuracy as part of \cref{sec:evaluation}. With the model-agnostic interpretability method \gls{SAGE}, we attribute predictions to features and cross-compare the feature importances of classical trade classification rules and machine learning predictors. In turn, for Transformers attention maps provide additional insights into the model. Lastly, \cref{sec:application} tests all classifiers in the problem of 
effective spread calculation to demonstrate the effectiveness of our approach.