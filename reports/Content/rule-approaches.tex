\section{Rule-Based Approaches}\label{sec:rule-based-approaches}

Every option trade has a buyer and seller side. For a plethora of problems in option research, it’s also crucial to determine the party that initiated the transaction.

In the following sections, we provide a concise overview of different trade initiator definitions and present rule-based methods for trade classification.

\subsection{Trade Initiator}
\label{sec:trade-initiator}

Various definitions for the trade initiator have been proposed in prior
research. Among these, the:

\emph{Chronological view:} \textcite[][267]{odders-whiteOccurrenceConsequencesInaccurate2000} adapts a chronological view based on the order arrival. She defines the initiator of the trade as the party (buyer or seller) who places their order last, chronologically. This definition requires knowledge about the order submission times.

\emph{Immediacy view:} In contrast, \textcite[][94--97]{leeInferringInvestorBehavior2000} equate the trade initiator with the party in demand for immediate execution. Thus, traders placing market orders, immediately executable at whatever price, or executable limit orders, are considered the trade initiator. By contrast, the party placing non-executable limit orders, which may not even result in a trade, is the non-initiator. This definition remains ambiguous for trades resulting from crossed limit orders, matched market orders, or batched orders \autocite[][94--95]{leeInferringInvestorBehavior2000}.
% FIXME: introduce of the notion of demanding/taking away liquidity and providing liquidity

\emph{Positional view:} Independent from the order type and submission time, \textcite[][533]{ellisAccuracyTradeClassification2000} deduce their definition of the trade initiator based on the position of the involved parties opposite to the market maker or broker. The assumption is, that the market maker or broker only provides liquidity to the investor and the trade would not exist without the initial investor's demand.

\todo{Quote rule is based on assumption that market maker provides quotes, which is neq to unexecuted limit orders of customers. Positional view refers to open/close dataset, where the indicator is available. Algorithms are applied to intraday data where the types (cust, mm, broker) can not be distinguished}

The appropriate view differs by data availability and application context.
Regardless of the definition used, the trade initiator is binary and can either be the seller or the buyer. Henceforth, we denote it by $\gls{y} \in \mathcal{Y}$ with $\mathcal{Y}=\{-1,1\}$, with $y=-1$ indicating a seller-initiated and $y=1$ a buyer-initiated trade. The predicted trade initiator is distinguished by $\hat{y}$.

\todo{Make sure it is clear, why the definitions were introduced. Explain how our definition fits into the construct and why it was chosen. Refer to later chapter.}

As the trade initiator is commonly not provided with option datasets, it must be inferred using trade classification algorithms \autocite[][453]{easleyOptionVolumeStock1998}. The following section introduces basic rules for trade classification. We start with the ubiquitous quote and tick rule and continue with the more recent depth and trade size rule. Our focus is on classification rules, that sign trades on a trade-by-trade basis. Consequently, we omit classification rules for aggregated trades, like the \gls{BVC} algorithm of \textcite[][1466--1468]{easleyFlowToxicityLiquidity2012}.

\subsection{Basic Rules}\label{sec:basic-rules}

This section presents basic classification rules, that may be used for trade classification independently or integrated into a hybrid

 algorithm.

\subsubsection{Quote Rule}\label{sec:quote-rule}

\todo{Add intuition. Classical reasoning is, market maker provides quotes. }

The quote rule classifies a trade by comparing the trade price against the corresponding quotes at the time of the trade. We denote the sequence of trade prices of the $i$-th security by $\gls{P}_i = \langle P_{i,1},P_{i,2},\dots,P_{i,T}\rangle$ and the corresponding ask at $t$ by $\gls{A}_{i,t}$ and bid by $\gls{B}_{i,t}$. If the trade price is above the midpoint of the bid-ask spread, estimated as $\gls{M}_{i,t} = \tfrac{1}{2}(B_{i,t} + A_{i,t})$, the trade is classified as a buy and if it is below the midpoint, as a sell \autocite[][41]{harrisDayEndTransactionPrice1989}. Thus, the classification rule on $\mathcal{A} = \left\{(i, t) \in \mathbb{N}^2: P_{i,t} \neq M_{i,t}\right\}$ is given by:
\begin{equation}
    \operatorname{quote}\colon \mathcal{A} \to \mathcal{Y},\quad
    \operatorname{quote}(i, t)=
    \begin{cases}
        1,  & \mathrm{if}\ P_{i, t}>M_{i, t}  \\
        -1, & \mathrm{if}\ P_{i, t}<M_{i, t}. \\
    \end{cases}
\end{equation}
By definition, the quote rule cannot classify trades at the midpoint of the quoted spread. \textcite[][241]{hasbrouckTradesQuotesInventories1988} discusses multiple alternatives for signing midspread trades including ones based on the subsequent quotes, and contemporaneous, or the subsequent transaction. Yet, the most common solution to overcome this limitation is, coupling the quote rule with other approaches, as done in \cref{sec:hybrid-rules}.

\todo{How are Missing quotes / midspreads handled
?}

The quote rule can be estimated at the exchange level or \gls{NBBO}.

\subsubsection{Tick Test}\label{sec:tick-test}

\todo{Add more intuition. Classical rational, buyer is willing to pay premium or seller is willing to sell at a premum. Thus, trade prices increase etc.}

A common alternative to the quote rule is the tick test. Based on the rationale that buys increase the trade price and sells lower them, the tick test classifies trades by the change in trade price. It was first applied in \textcites[][244]{holthausenEffectLargeBlock1987}[][240]{hasbrouckTradesQuotesInventories1988}. The tick test is defined as:
\begin{equation}
    \operatorname{tick}\colon \mathbb{N}^2 \to \mathcal{Y},\quad
    \operatorname{tick}(i, t)=
    \begin{cases}
        1,                           & \mathrm{if}\ t=1 \lor P_{t}>P_{t-1} \\
        -1,                          & \mathrm{if}\ P_{i, t} < P_{i, t-1}  \\
        \operatorname{tick}(i, t-1), & \mathrm{else}.
    \end{cases}
    \label{eq:tick-test}
\end{equation}
Considering the cases in \cref{eq:tick-test} the trade price is higher than the previous price (uptick) the trade is classified as a buy~\footnote{To end recursion at $t=1$, we sign the trade to be buyer-initiated to simplify notation. Other choices are possible, e.g., random assignment or based on another rule. Similarly done for \cref{eq:reverse-tick-test}.}. Reversely, if it is below the previous price (downtick), the trade is classified as a sell. If the price change is zero (zero tick), the signing uses the last price different from the current price \autocite[][735]{leeInferringTradeDirection1991}.

\todo{End of recursion complicates things. Think of random classification to allign with later or remove it.}

By this means, the tick rule can sign all trades as long as a last differing trade price exists, but the overall precision can be impacted by infrequent trading. Being only dependent on transaction data makes the tick rule highly data-efficient. Waiving any quote data for classification contributes to this efficiency, but also poses a major limitation with regard to trades at the bid or ask, as discussed by \textcite[][557--558]{finucaneDirectTestMethods2000}. For instance, if quotes rise between trades, then a sale at the bid on an uptick or zero uptick is misclassified as a buy by the tick test due to the overall increased trade price. Similarly for falling quotes, buys at the ask on downticks or zero downticks are erroneously classified as a sell.

The reverse tick test is a variant of the tick test proposed in \textcite[][241]{hasbrouckTradesQuotesInventories1988}. It is similar to the tick rule but classifies based on the next, distinguishable trade price.

\begin{equation}
    \operatorname{rtick} \colon \mathbb{N}^2 \to \mathcal{Y},\quad
    \operatorname{rtick}(i, t)=
    \begin{cases}
        1,                            & \mathrm{if}\ t+1=T \lor P_{i, t} > P_{i, t+1} \\
        -1,                           & \mathrm{if}\ P_{i, t} < P_{i, t+1}            \\
        \operatorname{rtick}(i, t+1), & \mathrm{else}
    \end{cases}
    \label{eq:reverse-tick-test}
\end{equation}
As denoted in \cref{eq:reverse-tick-test}, the trade is classified as seller-initiated, if the next trade is on an uptick or a zero uptick, and classified as buyer-initiated for trades at a downtick or a zero downtick \autocite[][735--636]{leeInferringTradeDirection1991}.

Both tests result in the same classification, if the current trade is bracketed by a price reversal and the price change after the trade is opposite from the change before the trade, but differ for price continuations when price changes are in the same direction \autocite[][736]{leeInferringTradeDirection1991}.

The performance of the tick rules hinges on the availability of recent trade prices for inference. For infrequently traded assets this can pose a problem, as outdated prices might loose their relevancy for the classification.

\subsubsection{Depth Rule}\label{sec:depth-rule}

As \cref{sec:quote-rule} discusses, the quote rule necessitates alternative procedures for midspread trades. For midspread trades, that otherwise cannot be classified by the quote rule, \textcite[][14]{grauerOptionTradeClassification2022} propose the depth rule as an override.

The depth rule gauges the trade initiator from the quoted size at the best bid and ask. Based on the observation that an exceeding bid or ask size relates to higher liquidity at one trade side, trades are classified as a buy (sell) for a larger ask (bid) size \autocite[][14]{grauerOptionTradeClassification2022}.

Let $\gls{A-tilde}_{i,t}$ denote the quoted size of the ask, $\gls{B-tilde}_{i,t}$ of the bid, and $P_{i,t}$ the trade Price at $t$ of the $i$-th option. We set the domain as $\mathcal{A} = \left\{(i, t) \in \mathbb{N}^2: P_{i,t} = \gls{M}_{i,t} \land \tilde{A}_{i,t} \neq \tilde{B}_{i,t} \right\}$. The depth rule is now calculated as:
\begin{equation}
    \operatorname{depth} \colon \mathcal{A} \to \mathcal{Y},\quad
    \operatorname{depth}(i, t)=
    \begin{cases}
        1,  & \mathrm{if}\ \tilde{A}_{i,t} > \tilde{B}_{i,t}. \\
        -1, & \mathrm{if}\ \tilde{A}_{i,t} < \tilde{B}_{i,t}\\
    \end{cases}
    \label{eq:depth-rule}
\end{equation}
As shown in \cref{eq:depth-rule}, the depth rule classifies midspread trades only, if the ask size is different from the bid size, as the ratio between the ask and bid size is the sole criterion for inferring the trade's initiator. Due to these restrictive conditions in $\mathcal{A}$, the depth rule can sign only a fraction of all trades and must be best stacked with other rules.

\subsubsection{Trade Size Rule}\label{sec:trade-size-rule}

% TODO: Think about writing as restriction? https://en.wikipedia.org/wiki/Restriction_(mathematics)
Generally, quote-based approaches are preferred due to their strong performance. \textcite[][13]{grauerOptionTradeClassification2022} stress, however, that the quote rule systematically misclassifies limit orders, and propose an override. The trade size rule is defined on $\mathcal{A} = \left\{(i, t) \in \mathbb{N}^2: \tilde{P}_{i,t} = \tilde{A}_{i,t} \neq \tilde{B}_{i,t} \lor \tilde{P}_{i,t} \neq\tilde{A}_{i,t} = \tilde{B}_{i,t} \right\}$ as:
\begin{equation}
    \operatorname{tsize} \colon \mathcal{A} \to \mathcal{Y},\quad
    \operatorname{tsize}(i, t)=
    \begin{cases}
        1,  & \mathrm{if}\ \tilde{P}_{i, t} = \tilde{B}_{i, t} \neq \tilde{A}_{i, t}  \\
        -1, & \mathrm{if}\ \tilde{P}_{i, t} = \tilde{A}_{i, t} \neq \tilde{B}_{i, t}. \\
    \end{cases}
    \label{eq:trade-size-rule}
\end{equation}
The trade size rule in \cref{eq:trade-size-rule} classifies based on a match between the size of the trade $\tilde{P}_{i, t}$ and the quoted bid and ask sizes. The rationale is, that the market maker tries to fill the limit order of a customer, which results in the trade being executed at the contemporaneous bid or ask, with a trade size equalling the quoted size \autocite[][13]{grauerOptionTradeClassification2022}. When both the size of the ask and bid correspond with the trade size or the trade size does not match the quoted sizes, the result is ambiguous.

Expectedly, the improvement is highest for trades at the quotes and reverses for trades outside the quote \autocite[][15]{grauerOptionTradeClassification2022}. Based on these results, the trade size rule may only be applied selectively to trades near or at the quote. Since only a fraction of all trades can be classified with the trade size rule, the rule must be combined with other basic or hybrid rules for complete coverage. The subsequent section introduces four hybrid algorithms, that combine basic rules into more sophisticated algorithms.

\subsection{Hybrid Rules}\label{sec:hybrid-rules}

The basic trade classification rules from \cref{sec:basic-rules} can be combined into a hybrid algorithm to enforce universal applicability to all trades and improve the classification performance.

\begin{figure}[ht!]
    \hfill
    \subfloat[\acrshort{LR} Algorithm\label{fig:hybrid-lr}]{
        {\renewcommand\normalsize{\tiny}
                \normalsize
                \input{./Graphs/lr-algo.pdf_tex}}
    }
    \subfloat[\acrshort{EMO} Rule\label{fig:hybrid-emo}]{
        {\renewcommand\normalsize{\tiny}
                \normalsize
                \input{./Graphs/emo-algo.pdf_tex}}
    }
    \subfloat[\acrshort{CLNV} Rule\label{fig:hybrid-clnv}]{
        {\renewcommand\normalsize{\tiny}
                \normalsize
                \input{./Graphs/clnv-algo.pdf_tex}}
    }
    \subfloat[Stacked Rule\label{fig:hybrid-grauer}]{
        {\renewcommand\normalsize{\tiny}
                \normalsize
                \input{./Graphs/grauer-algo.pdf_tex}}
    }
    \hfill\null
    \caption[Comparison Between Hybrid Trade Classification Rules]{Comparison between hybrid trade classification rules. The Figure visualises the components of the \acrshort{LR} algorithm, \acrshort{EMO} rule, the \acrshort{CLNV} method, and an arbitrary, stacked combination relative to the quotes. Rules at the midpoint or the quotes are slightly exaggerated for better readability. Own work inspired by \textcite[][167]{poppeSensitivityVPINChoice2016}.}
    \label{fig:hybrid-algorithms}
\end{figure}

Popular variants include the \gls{LR} algorithm, the \gls{EMO} rule, and the \gls{CLNV} method. All three algorithms utilise the quote and tick rule to a varying extent, as depicted in \cref{fig:hybrid-lr,fig:hybrid-emo,fig:hybrid-clnv}. Basic rules are selected based on the proximity of the trade price to the quotes. We study all algorithms in detail in \cref{sec:lee-and-ready-algorithm,sec:ellis-michaely-ohara-rule,sec:chakarabarty-li-nguyen-van-ness-method}.


As put forth by \textcite[][18]{grauerOptionTradeClassification2022}, basic or hybrid rules can be combined through stacking. One such combination is depicted in \cref{fig:hybrid-grauer}. This approach generalises the aforementioned algorithms, as the applied rule is no longer dependent on the proximity to the quotes, but rather on the classifiability of the trade with the primary rules given by the domains and their ordering. We cover this approach last.

\subsubsection{Lee and Ready Algorithm}\label{sec:lee-and-ready-algorithm}

The popular \gls{LR} algorithm \autocite[][745]{leeInferringTradeDirection1991} combines the (reverse) tick test and quote rule into a single rule, which is derived from two observations. First, \textcite[][735--743]{leeInferringTradeDirection1991} observe a higher precision of the quote rule over the tick rule, which makes it their preferred choice. Second, by the means of a simple model, the authors demonstrate that the tick test can correctly classify at least \SI{85.0}{\percent} of all midspread trades if the model's assumptions of constant quotes between trades and the arrival of the market and standing orders following a Poisson process are met.

In combination, the algorithm primarily signs trades according to the quote rule. Trades at the midpoint of the spread, unclassifiable by the quote rule, are classified by the tick rule. Overall:
\begin{equation}
    \operatorname{lr} \colon \mathbb{N}^2 \to \mathcal{Y},\quad\operatorname{lr}(i,t)=
    \begin{cases}
        1,                         & \mathrm{if}\ P_{i, t} > M_{i, t} \\
        -1,                        & \mathrm{if}\ P_{i, t} < M_{i, t} \\
        \operatorname{tick}(i, t), & \mathrm{else}.
    \end{cases}
\end{equation}
As the algorithm requires both trade and quote data, it is less data-efficient than its subparts. Even if data is readily available, in past option studies the algorithm does not significantly outperform the quote rule and outside the model's tight assumptions the expected accuracy of the tick test is unmet \autocites[][30--32]{grauerOptionTradeClassification2022}[][886]{savickasInferringDirectionOption2003}. Nevertheless, the algorithm is a common choice in option research \autocite[cp.][453]{easleyOptionVolumeStock1998}. It is also the basis for more advanced algorithms, such as the \gls{EMO} rule, which is next.

\subsubsection{Ellis-Michaely-O'Hara
    Rule}\label{sec:ellis-michaely-ohara-rule}

\textcite[][536]{ellisAccuracyTradeClassification2000} examine the performance of the previous algorithms for stocks traded at \gls{NASDAQ}. By analysing miss-classified trades with regard to the proximity of the trade to the quotes, they observe, that the quote rule and by extension, the \gls{LR} algorithm, perform particularly well at classifying trades executed at the bid and the ask price but trail the performance of the tick rule for trades inside or outside the spread \autocite[][535--536]{ellisAccuracyTradeClassification2000}. The authors combine these observations into a single rule, known as the \gls{EMO} algorithm.

As such, the \gls{EMO} algorithm extends the tick rule by classifying trades at the quotes using the quote rule, and all other trades with the tick test. Formally, the classification rule is given by:
\begin{equation}
    \operatorname{emo} \colon \mathbb{N}^2 \to \mathcal{Y}, \quad
    \operatorname{emo}(i, t)=
    \begin{cases}
        1,                         & \mathrm{if}\ P_{i, t} = A_{i, t} \\
        -1,                        & \mathrm{if}\ P_{i, t} = B_{i, t} \\
        \operatorname{tick}(i, t), & \mathrm{else}.
    \end{cases}
    \label{eq:emo-rule}
\end{equation}
\Cref{eq:emo-rule} embeds both the quote and tick rule. As trades off the quotes are classified by the tick rule, the algorithm's overall success rate is dominated by the tick test assuming most trades are off-the-quotes. For option markets \autocites[cp.][891]{savickasInferringDirectionOption2003}[][21]{grauerOptionTradeClassification2022} this dependence causes the performance to lag behind quote-based approaches, contrary to the successful adaption in the stock market \autocites[][541]{ellisAccuracyTradeClassification2000}[][3818]{chakrabartyTradeClassificationAlgorithms2007}. \textcite[][31--35]{grauerOptionTradeClassification2022} improve the classification accuracy for option trades by applying the reverse tick test as a proxy for the tick test.

\subsubsection{Chakrabarty-Li-Nguyen-Van-Ness
    Method}\label{sec:chakarabarty-li-nguyen-van-ness-method}

Like the previous two algorithms, the \gls{CLNV} method of \textcite[][3809]{chakrabartyTradeClassificationAlgorithms2012} is a hybrid of the quote and tick rule and extends the \gls{EMO} rule by a differentiated treatment of trades inside the quotes, which are notoriously hard to classify. The authors segment the bid-ask spread into deciles (ten equal-width bins) and classify trades around the midpoint (fourth to seventh decile) by the tick rule and trades close or outside the quotes are categorised by the tick rule.
\begin{equation}
    \operatorname{clnv} \colon \mathbb{N}^2 \to \mathcal{Y}, \quad
    \operatorname{clnv}(i, t)=
    \begin{cases}
        1,                         & \mathrm{if}\ P_{i, t} \in \left(\frac{3}{10} B_{i,t} + \frac{7}{10} A_{i,t}, A_{i, t}\right] \\
        -1,                        & \mathrm{if}\ P_{i, t} \in \left[ B_{i,t}, \frac{7}{10} B_{i,t} + \frac{3}{10} A_{i,t}\right) \\
        \operatorname{tick}(i, t), & \mathrm{else}
    \end{cases}
    \label{eq:CLNV-rule}
\end{equation}

The algorithm is summarised in \cref{eq:CLNV-rule}. It is derived from a performance comparison of the tick rule (\gls{EMO} rule) against the quote rule (\gls{LR} algorithm) on stock data, whereby the accuracy was assessed separately for each decile \footnote{The spread is assumed to be positive and evenly divided into ten deciles and the first to third deciles are classified by the quote rule. Counted from the bid, the first decile starts at $B_{i,t}$ and ends at $B_{i,t} + \tfrac{3}{10} (A_{i,t} - B_{i,t}) = \tfrac{7}{10} B_{i,t} + \tfrac{3}{10} A_{i,t}$ third decile. As all trade prices are below the midpoint, they are classified as a sell.}. The classical \gls{CLNV} method uses the backward-looking tick rule. In the spirit of \textcite[][735]{leeInferringTradeDirection1991}, the tick test can be exchanged for the reverse tick test.

\subsubsection{Stacked Rule}\label{sec:stacked-rule}

The previous algorithms are static concerning the used base rules and their alignment. Combining arbitrary rules into a single algorithm requires a generic procedure. \textcite[][18]{grauerOptionTradeClassification2022} combine basic and hybrid rules through stacking. In this setting, the trade traverses a stack of pre-defined rules until a rule can classify the trade or the end of the stack is reached~\footnote{For a trade, which cannot be classified by any classifier, one may fallback on a random assignment or the majority class if the distribution of trades is imbalanced.}. The classification is now dependent on the employed rules but also on their relative ordering.

The most basic application is in the \gls{LR} algorithm, combining $\operatorname{quote} \to \operatorname{tick}$, whereby the quote rule is applied first as indicated by the arrow. For a more complex example consider the hybrid rule consisting of $\operatorname{tsize}_{\mathrm{ex}} \to \operatorname{quote}_{\mathrm{nbbo}} \to \operatorname{quote}_{\mathrm{ex}} \to \operatorname{depth}_{\mathrm{nbbo}} \to \operatorname{depth}_{\mathrm{ex}} \to \operatorname{rtick}_{\mathrm{all}}$ popularized in \textcite[][18]{grauerOptionTradeClassification2022}. We refer to this specific combination as the \gls{GSU} rule. Only a fraction of all trades are classifiable by the trade size rule, which is the primary rule, due to a narrow domain, and classification is deferred to lower rules in the stack, specifically the quote rule at the \gls{NBBO}, which by design has larger coverage. Theoretically, stacked rules can grow to great depth with an arbitrary arrangement. In practice, rules may be ordered greedily and new rules added if there are unclassified trades.

\textcite[][3811]{chakrabartyTradeClassificationAlgorithms2007} and \textcite[][18]{grauerOptionTradeClassification2022} continue the trend for more complex classification rules, leading to a higher fragmented decision surface, and eventually resulting in improved classification accuracy. Since the condition, for the selection of the base rule, is inferred from \emph{static} cut-off points at the decile boundaries of the spread including the midspread and quotes. This raises the question of whether classifiers trained on price and quote data can adapt to the data and improve upon classical trade classification rules.

The trend towards sophisticated, hybrid rules, combining as many as six base rules into a single classifier, has conceptual parallels to stacked ensembles found in machine learning and expresses the need for better classifiers.

We provide an overview of state-of-the-art machine learning-based classifiers and start by framing trade classification as a supervised learning
problem.