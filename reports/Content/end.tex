\section{Conclusion and Outlook}\label{sec:conclusion-outlook}

The goal of this study is to examine the performance of machine learning-based trade classification in the option market. In particular, we propose to model trade classification with Transformers and gradient boosting. Both approaches are supervised and leverage labeled trades. For settings where labeled trades are scarce, we extend Transformers with a pre-training objective to train on unlabeled trades as well as generate pseudo labels for gradient boosting through a self-training procedure.

Our models establish a new state-of-the-art for trade classification on the \gls{ISE} and \gls{CBOE} dataset. For \gls{ISE} trades, Transformers achieve an accuracy of \SI{63.78}{\percent} when trained on trade and quoted prices as well as \SI{72.58}{\percent} when trained on additional quoted sizes, improving over current best of \textcite[][15]{grauerOptionTradeClassification2022} by \SI{3.73}{\percent} and \SI{4.97}{\percent}. Similarly, \glspl{GBRT} reach accuracies between \SI{63.67}{\percent} and \SI{72.34}{\percent}. We observe performance improvements up to \SI{6.51}{\percent} for \glspl{GBRT} and \SI{6.31}{\percent} for Transformers when models have access to option characteristics. Relative to the ubiquitous tick test, quote rule, and \gls{LR} algorithm, improvements are \SI{23.88}{\percent}, \SI{17.11}{\percent}, and \SI{17.02}{\percent}. Outperformance is particularly strong for \gls{ITM} options, options with a long maturity, as well as options traded at the quotes. Both architectures generalize well on \gls{CBOE} data, with even stronger improvements between \SI{5.26}{\percent} and \SI{7.86}{\percent} over the benchmark depending on the model and feature set. 

In the semi-supervised setting, Transformers on \gls{ISE} dataset profit from pre-training on unlabeled trades with accuracies up to \SI{74.55}{\percent}, but the performance gains slightly diminish on the \gls{CBOE} test set. Vice versa, we observe no benefits from semi-supervised training of \glspl{GBRT}.

Unlike previous studies, we can trace back the performance of our approaches as well as of trade classification rules to individual features and feature groups using the importance measure \gls{SAGE}. We find that both paradigms attain the largest performance improvements from classifying trades based on quoted sizes and prices, but machine learning-based classifiers attain higher performance gains and effectively exploit the data. The change in the trade price, decisive criteria to the (reverse) tick test, plays no role in option trade classification. Our classifiers profit from the inclusion of option-specific features, like moneyness and time to maturity, currently unexploited in classical trade classification.

By probing and visualizing the attention mechanism of the Transformer, we can establish a connection to rule-based classification. Graphically, our results show, that attention heads encode knowledge about rule-based classification. Whilst attention heads in earlier layers of the network broadly attend to all features or their embeddings, later they focus on specific features jointly used in rule-based classification akin to the \gls{LR} algorithm, depth rule or others. Furthermore, embeddings encode domain knowledge. Our results demonstrate exemplary for the traded underlying, that the Transformer learns to group similar underlyings in embedding space.

Our classifiers deliver accurate predictions and improved robustness, which effectively reduces noise and bias in option research dependent on reliable trade initiator estimates. When applied to measuring trading cost through effective spreads, the models dominate all rule-based approaches by approximating the true effective spread of options best. 

In conclusion, our study showcases the efficacy of machine learning as a viable alternative to existing trade signing algorithms for classifying option trades, if partially-labeled or labeled trades are available for training.

In future work, we plan to revisit training Transformers on a larger corpus of unlabeled trades through pre-training objectives and study the effects from \emph{exchange-specific} fine-tuning. While our current results show that pre-training positively drives classification performance, for comparability it is only performed on a small subset of trades and models have not fully converged. Thus, we expect to see benefits from additional data and compute, following the scaling laws of \textcite[][30022]{hoffmannTrainingComputeOptimalLarge2022}. The application confers advantages when fine-tuning is constrained due to the limited availability of the true trade initiator.

Indicatively, our results show that specific attention heads in the Transformer specialize in patterns akin to classical trade classification rules. We want to explore this aspect further and potentially reverse engineer classification rules from attention heads that are yet unknown. This way, we can transfer the superior classification accuracy of the Transformer to regimes where labels are unavailable or computational costs of training are not affordable.