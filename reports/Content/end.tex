\section{Discussion}\label{sec:discussion}

Relative to related works performing trade classification using machine learning, the improvements are strong, as a comparison against \cref{app:literature-ml-tc} reveals.

\newpage
\section{Conclusion}\label{sec:conclusion}

Our models deliver accurate predictions and improved robustness, which effectively reduces noise and bias in option research dependent on reliable trade initiator estimates. When applied to measuring trading cost through effective spreads, the models dominate all rule-based approaches by approximating the true effective spread of options best. Concretely, the Transformer pre-trained on unlabelled trades estimates a mean spread of  \SI[round-mode=places, round-precision=3]{0.013118}[\$]{} versus \SI[round-mode=places, round-precision=3]{0.004926}[\$]{} actual spread at the \gls{ISE}.

In conclusion, our study showcases the efficacy of machine learning as a viable alternative to existing trade signing algorithms for classifying option trades, if partially-labelled or labelled trades are available for training. 

\newpage
\section{Outlook}\label{sec:outlook}

Graphically, our results show that specific attention heads in the Transformer specialise in patterns akin to classical trade classification rules. We are excited to explore this aspect systematically and potentially reverse engineer classification rules from attention heads that are yet unknown. This way, we can transfer the superior classification accuracy of the Transformer to regimes where labelled training data is abundant or computational costs of training are not affordable.