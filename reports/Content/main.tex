\section{Introduction (2~p)}\label{introduction}

The authors acknowledge support by the state of Baden-Württemberg through \href{https://www.bwhpc.de/}{bwHPC}.

\section{Related Work (3~p)}\label{related-work}

\newpage
\section{Rule-Based Approaches (5.75~p)}\label{rule-based-approaches}


The following section introduces common rules for signing option trades. We start by introducing the classical quote and tick rule and continue with the more recent depth and trade size rule. In section \ref{hybrid-rules} we combine some rules from section \ref{basic-rules} to hybrids thereof. We conclude this chapter by drawing a connection to ensemble learning.

\subsection{Basic Rules (2.5~p)}\label{basic-rules}

Starting with the quote rule, we describe the most common rule for signing option trades, which can either be used as-is or joint in more complex rules.

\subsubsection{Quote-Rule (0.5~p)}\label{quote-rule}

\begin{algorithm}

  % input/ouput names
  \SetKwInOut{Input}{Input}
  \SetKwInOut{Output}{Output}

  % caption
  % TODO: set input and output: e. g., $\hat{e} \leftarrow$ layer_norm $(e \mid \gamma, \beta)$
  \caption{$\operatorname{\mathtt{quote}}$ \label{alg:quote-rule}}

  \Input{%
    $t_i$ trade price at $i$, $a_i$ ask price at $i$, and $b_i$ bid price at $i$.
  }
  \Output{%
    $o_i \in\{-1,1\}$ trade initiator for $i$-th trade.
  }

  \BlankLine % blank line for spacing

  % start of the pseudocode
  $m_i \leftarrow \frac{1}{2}(a_i + b_i)$ \tcc*{mid spread at $i$}

  \uIf{$t_i > m_i$}{%
    \Return{$o_i =1$}
  }
  \uElseIf{$t_i < m_i$}{%
    \Return{$o_i =-1$}
  }
  \uElse{%
    \Return
  }
\end{algorithm}




\subsubsection{Tick Test (1~p)}\label{tick-test}



% \subsubsection{Reverse Tick Test (0.5~p)}\label{reverse-tick-test}

\subsubsection{Depth Rule (0.5~p)}\label{depth-rule}

\textcite{grauerOptionTradeClassification2022} promote an alternative to improve the classification performance of midspread trades. In their \textit{depth rule}, they infer the trade initiator from the depth of the ask and bid. Based on the observation that an exceeding bid or ask size relates to higher liquidity on one side, trades are classified as buyer-initiated for a larger ask size and seller-initiated for a higher bid size.

As shown in Algorithm \ref{alg:depth-rule}, the depth rule classifies midspread trades only, if the ask size differs from the bid size, as the ratio between the ask and bid size is the sole criterion for assigning the initiator. To sign the remaining trades, other rules must be employed thereafter.

\begin{algorithm}

  % input/ouput names
  \SetKwInOut{Input}{Input}
  \SetKwInOut{Output}{Output}

  % caption
  % TODO: set input and output: e. g., $\hat{e} \leftarrow$ layer_norm $(e \mid \gamma, \beta)$
  \caption{$\operatorname{\mathtt{depth}}$ \label{alg:depth-rule}}

  \Input{%
    $t_i$ trade price at $i$, $a_i$ ask price at $i$, $b_i$ bid price at $i$, $\tilde{a}_i$ ask size at $i$, and $\tilde{b}_i$ bid size at $i$.
  }
  \Output{%
    $o_i \in\{-1,1\}$ trade initiator for $i$-th trade.
  }

  \BlankLine % blank line for spacing

  % start of the pseudocode
  $m_i \leftarrow \frac{1}{2}(a_i + b_i)$ \tcc*{mid spread at $i$}

  \uIf{$t_i = m_i$}{
    \uIf{$\tilde{a}_i > \tilde{b}_i$}{
      \Return{$o_i =1$}
    }
    \uElseIf{$\tilde{a}_i < \tilde{b}_i$}{
      \Return{$o_i =-1$}
    }
    \uElse{
      \Return
    }
  }
  \uElse{
    \Return \tcc*{apply secondary rule}
  }
\end{algorithm}

In a similar vein, the \textit{trade size rule} reuses the ask and bid quote size to improve the classification performance of trades where the trade size equals the ask quote or bid quote sizes.

\subsubsection{Trade Size Rule (0.5~p)}\label{trade-size-rule}

\begin{algorithm}

  % input/ouput names
  \SetKwInOut{Input}{Input}
  \SetKwInOut{Output}{Output}
  \SetKw{And}{\textbf{and}}
  % caption
  % TODO: set input and output: e. g., $\hat{e} \leftarrow$ layer_norm $(e \mid \gamma, \beta)$
  \caption{$\operatorname{\mathtt{tradesize}}(t_i, a_i, b_i)$ \label{alg:tradesize-rule}}

  \Input{%
    $\tilde{t}_i$ trade size at $i$, $\tilde{a}_i$ ask size at $i$, and $\tilde{b}_i$ bid size at $i$.
  }
  \Output{%
    $o_i \in\{-1,1\}$ trade initiator for $i$-th trade.
  }

  \BlankLine % blank line for spacing

  \uIf{$\tilde{a}_i = \tilde{t}_i$ \And $\tilde{b}_i \neq \tilde{t}_i$}{%
    \Return{$o_i =-1$}
  }
  \uElseIf{$\tilde{b}_i = \tilde{t}_i$ \And $\tilde{a}_i \neq \tilde{t}_i$}{%
    \Return{$o_i =1$}
  }
  \uElse{%
    \Return \tcc*{apply secondary rule}
  }
\end{algorithm}


\subsection{Hybrid Rules (3.25~p)}\label{hybrid-rules}

\subsubsection{Lee and Ready Algorithm (1 p)}\label{lee-and-ready-algorithm}

\begin{algorithm}

  % input/ouput names
  \SetKwInOut{Input}{Input}
  \SetKwInOut{Output}{Output}

  % caption
  % TODO: set input and output: e. g., $\hat{e} \leftarrow$ layer_norm $(e \mid \gamma, \beta)$
  \caption{$\operatorname{\mathtt{lee-ready}}{(t_i, a_i, b_i)}$ \label{alg:lee-ready-algorithm}}

  \Input{%
    $t_i$ trade price at $i$, $a_i$ ask price at $i$, and $b_i$ bid price at $i$.
  }
  \Output{%
    $o_i \in\{-1,1\}$ trade initiator at $i$. \\
  }

  \BlankLine % blank line for spacing

  % start of the pseudocode
  $m_i \leftarrow \frac{1}{2}(a_i + b_i)$ \tcc*{mid spread at $i$}

  \For{$1, \cdots, I$}{
    \uIf{$t_i > m_i$}{
      \Return{$o_i = 1$}
    }
    \uElseIf{$t_i < m_i$}{
      \Return{$o_i = -1$}
    }
    \Else{
      \Return{$o_i = \operatorname{\mathtt{tick}}{(t_i, a_i, b_i)}$} \tcc*{see above}
    }
  } % end for i
  % TODO: set input and output params
\end{algorithm}


% \subsubsection{Reverse Lee and Ready
%   Algorithm (0.5~p)}\label{reverse-lee-and-ready-algorithm}

\subsubsection{Ellis-Michaely-O'Hara
  Rule (0.5~p)}\label{ellis-michaely-ohara-rule}

\begin{algorithm}

  % input/ouput names
  \SetKwInOut{Input}{Input}
  \SetKwInOut{Output}{Output}

  % caption
  % TODO: set input and output: e. g., $\hat{e} \leftarrow$ layer_norm $(e \mid \gamma, \beta)$
  \caption{$\operatorname{\mathtt{emo}}$ \label{alg:emo-rule}}

  \Input{%
    $t_i$ trade price at $i$, $a_i$ ask price at $i$, and $b_i$ bid price at $i$.
  }
  \Output{%
    $o_i \in\{-1,1\}$ trade initiator at $i$.
  }

  \BlankLine % blank line for spacing

  % start of the pseudocode
  \For{$1, \cdots, I$}{
    \uIf{$t_i = a_i$}{
      \Return{$o_i = 1$}
    }
    \uElseIf{$t_i = b_i$}{
      \Return{$o_i = -1$}
    }
    \Else{
      \Return{$o_i = \operatorname{\mathtt{tick}}{(t_i, a_i, b_i)}$} \tcc*{see above}
    }
  } % end for i
  % TODO: set input and output params
\end{algorithm}

\subsubsection{Chakrabarty-Li-Nguyen-Van-Ness
  Method (0.5~p)}\label{chakarabarty-li-nguyen-van-ness-method}


\subsubsection{Rosenthal's Rule (0.75~p)}\label{rosenthals-rule}

\newpage
\section{Supervised Approaches (12~p)}\label{supervised-approaches}

\subsection{Selection of Approaches (2~p)}\label{selection-of-approaches}

\subsection{Gradient Boosted Trees (2~p)}\label{gradient-boosted-trees}

\subsubsection{Decision Tree (0.5~p)}\label{decision-tree}

\subsubsection{Gradient Boosting
  Procedure (1 p)}\label{gradient-boosting-procedure}

\subsubsection{Adaptions for Probabilistic
  Classification (0.5~p)}\label{adaptions-for-probablistic-classification}

\subsection{Transformer Networks (8 p)}\label{transformer-networks}

\subsubsection{Network Architecture (2.5~p)}\label{network-architecture}

\subsubsection{Attention (0.5~p)}\label{attention}

\subsubsection{Positional Encoding (0.5~p)}\label{positional-encoding}

\subsubsection{Embeddings (0.5~p)}\label{embeddings}

\subsubsection{Extensions in TabNet (2~p)}\label{extensions-in-tabnet}

\subsubsection{Extensions in
  TabTransformer (2~p)}\label{extensions-in-tabtransformer}

\newpage
\section{Semi-Supervised Approaches (8~p)}\label{semi-supervised-approaches}

\subsection{Selection of Approaches (2~p)}\label{selection-of-approaches-1}

\subsection{Extensions to Gradient Boosted
  Trees (2~p)}\label{extensions-to-gradient-boosted-trees}

\subsection{Extensions to TabNet (2~p)}\label{extensions-to-tabnet}

\subsection{Extensions to
  TabTransformer (2~p)}\label{extensions-to-tabtransformer}

\newpage
\section{Empirical Study (19.5~p)}\label{empirical-study}

\subsection{Environment (0.5~p)}\label{environment}

\subsection{Data and Data Preparation (6 p)}\label{data-and-data-preparation}

\subsubsection{ISE Data Set (0.5~p)}\label{ise-data-set}

\subsubsection{CBOE Data Set (0.5~p)}\label{cboe-data-set}

\subsubsection{Generation of True
  Labels (0.5~p)}\label{generation-of-true-labels}

\subsubsection{Feature Engineering (4~p)}\label{feature-engineering}

\subsubsection{Train-Test Split (0.5~p)}\label{train-test-split}

\subsection{Training and Tuning (10~p)}\label{training-and-tuning}

\subsubsection{Training of Supervised
  Models (4~p)}\label{training-of-supervised-models}


\subsubsection{Training of Semi-Supervised
  Models (4~p)}\label{training-of-semi-supervised-models}


\subsubsection{Hyperparameter Tuning (2~p)}\label{hyperparameter-tuning}


\subsection{Evaluation (3~p)}\label{evaluation}

\subsubsection{Feature Importance
  Measure (2~p)}\label{feature-importance-measure}

\subsubsection{Evaluation Metric (1~p)}\label{evaluation-metric}

\newpage
\section{Results (12~p)}\label{results}

\subsection{Results of Supervised
  Models (3~p)}\label{results-of-supervised-models}

\subsection{Results of Semi-Supervised
  Models (3~p)}\label{results-of-semi-supervised-models}

\subsection{Feature Importance (3~p)}\label{feature-importance}

\subsection{Robustness Checks (3~p)}\label{robustness-checks}

\newpage
\section{Discussion (3~p)}\label{discussion}

\newpage
\section{Conclusion (2~p)}\label{conclusion}

\newpage
\section{Outlook (0.5~p=67.75~p)}\label{outlook}

