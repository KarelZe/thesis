\section{Introduction}\label{sec:introduction}

Every option trade has a buyer and seller side. For a plethora of problems in option research, it’s also crucial to determine the party that initiated the transaction.  Common applications include the study of option demand \autocite[][3]{garleanuDemandBasedOptionPricing2009}, the informational content of option trading \autocites[][631]{huDoesOptionTrading2014}[][882]{panInformationOptionVolume2006} \todo{pin, effective spread, ...}.

Despite the clear importance for empirical research, the true initiator of the trade is frequently missing in option data sets and must be inferred using trade classification algorithms \autocite[][453]{easleyOptionVolumeStock1998}. In consequence, the correctness of empirical studies hinges on the algorithm's ability to correctly identify the trade initiator.

Among the most prevailing variants to sign trades are the tick rule \autocite[][240]{hasbrouckTradesQuotesInventories1988}, quote rule \autocite[][41]{harrisDayEndTransactionPrice1989}, and hybrids thereof such as the \gls{LR} algorithm \autocite[][745]{leeInferringTradeDirection1991}, the \gls{EMO} algorithm \autocite[][536]{ellisAccuracyTradeClassification2000}, and the \gls{CLNV} method \autocite[][3809]{chakrabartyTradeClassificationAlgorithms2007}, that infers the trade initiator from adjacent prices and quotes. These algorithms have initially been proposed and tested for the stock market.

For option markets, the works of \textcites[][10--13]{grauerOptionTradeClassification2022}[][887]{savickasInferringDirectionOption2003} raise concerns about the transferability of standard trade signing rules due to deteriorating classification accuracies and systematic miss-classification of trades.  \todo{Trade classification in option markets is a particularly difficult testing ground due to illiquidity / trading at different venues / \url{https://www.sec.gov/news/studies/ordpay.htm}. Enforce why these reasons make it even more important to have a good trade classification algorithm.}

Recent work of \textcite[][13--16]{grauerOptionTradeClassification2022} partly alleviates the concern by proposing explicit overrides for trade types and by combining multiple heuristics into deep-stacked rules, advancing the state-of-the-art performance in option trade classification. By this means, their approach enforces a more sophisticated decision boundary eventually leading to a more accurate classification. Beyond heuristics, however, it remains an open research problem in option markets, if classifiers \emph{learned} on trade data can improve upon static classification rules with respect to performance.

In this thesis, we focus on state-of-the-art machine learning methods to infer the trade initiator.\footnote{The authors acknowledge support by the state of Baden-Württemberg through \href{https://www.bwhpc.de/}{bwHPC}.} Approaching trade classification with machine learning is a logical choice, given its capability to handle high-dimensional trade data and learn complex decision boundaries. Against this backdrop, the question is, can an alternative machine learning-based classifier improve upon standard trade classification rules?

The remainder of this paper is organized as follows. \cref{sec:related-work} reviews publications on trade classification in option markets and using machine learning, thereby underpinning our research framework. \cref{sec:supervised-approaches} discusses and introduces supervised methods for trade classification. Then, \cref{sec:semi-supervised-approaches} extends the previously selected algorithms for the semi-supervised case. We test the models in \cref{sec:empirical-study} in an empirical setting. In \cref{sec:application} we apply our models to the problem of effective spread estimation. Finally, \cref{sec:discussion} discusses and \cref{sec:conclusion} concludes.
