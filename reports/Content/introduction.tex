\section{Introduction}\label{sec:introduction}

Every option trade has a buyer and seller side. For a plethora of problems in option research, it’s also crutial to determine the party that initiated the transaction. Common applications include the study of option demand \autocite[][3]{garleanuDemandBasedOptionPricing2009}, the informational content of option trading \autocites[][631]{huDoesOptionTrading2014}[][882]{panInformationOptionVolume2006}, of order flow \autocite[][684]{muravyevOrderFlowExpected2016}, or trading costs \autocite[][4980]{muravyevOptionsTradingCosts2020}. 
% \todo{informational content \autocite{easleyOptionVolumeStock1998}, pin, effective spread, ... An incomplete list of trading on the underlying is Easley et al. (1998), Pan and Poteshman (2006), Chen et al. (2018), Liu et al. (2017), Chordia et al.(2021), and on the volatility is Ni et al. (2008), Chang et al. (2010), Puhan (2014), Rourke (2014), Figlewski and Frommherz (2017), Kao et al. (2018), Lin et al. (2018), Ryu and Yang (2019).}.

Despite the clear importance for empirical research, the true initiator of the trade is frequently missing in option data sets and must be inferred using trade classification rules \autocite[][453]{easleyOptionVolumeStock1998}. In consequence, the correctness of empirical studies hinges on the algorithm's ability to accurately identify the trade initiator.

Among the most prevailing variants to sign trades are the tick test \autocite[][240]{hasbrouckTradesQuotesInventories1988}, quote rule \autocite[][41]{harrisDayEndTransactionPrice1989}, and hybrids thereof such as the \gls{LR} algorithm \autocite[][745]{leeInferringTradeDirection1991}, the \gls{EMO} algorithm \autocite[][536]{ellisAccuracyTradeClassification2000}, and the \gls{CLNV} method \autocite[][3809]{chakrabartyTradeClassificationAlgorithms2007}, that infer the trade initiator from adjacent prices and quotes. These heuristics have initially been proposed and tested in the stock market.

For option markets, the works of \textcites[][10--13]{grauerOptionTradeClassification2022}[][887]{savickasInferringDirectionOption2003} raise concerns about the transferability of standard trade signing rules due to deteriorating classification accuracies and systematic misclassifications. The latter is particularly problematic, as non-random misclassifications ultimately bias the research performed on top \autocite[][260]{odders-whiteOccurrenceConsequencesInaccurate2000}.

Recent work of \textcite[][13--16]{grauerOptionTradeClassification2022} partly alleviates the concern by proposing explicit overrides for trade types and by combining multiple heuristics into deep-stacked hybrids, thereby advancing the state-of-the-art performance in option trade classification. By this means, their approach enforces a more sophisticated decision boundary eventually leading to a more accurate classification. Beyond heuristics, however, it remains open, if classifiers \emph{learned} on trade data can improve upon \emph{static} classification rules in terms of performance and robustness.

Our work fills this gap by focusing on machine learning methods to infer the trade initiator in the option market.\footnote{The authors acknowledge support by the state of Baden-Württemberg through \href{https://www.bwhpc.de/}{bwHPC}.} Approaching trade classification with machine learning is a logical choice, given its capability to handle high-dimensional trade data and learn complex decision boundaries. Against this backdrop, the question is can an alternative machine learning-based classifier improve upon the accuracy of state-of-the-art approaches for option trade classification?

To answer this question, we (...). 

The remainder of this paper is organized as follows. \cref{sec:related-work} reviews publications on trade classification in option markets and using machine learning, thereby underpinning our research framework. \cref{sec:supervised-approaches} discusses and introduces supervised methods for trade classification. Then, \cref{sec:semi-supervised-approaches} extends the previously selected algorithms for the semi-supervised case. We test the models in \cref{sec:empirical-study} in an empirical setting. In \cref{sec:application} we apply our models to the problem of effective spread estimation. Finally, \cref{sec:discussion} discusses and \cref{sec:conclusion} concludes.
